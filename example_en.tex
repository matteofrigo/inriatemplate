%  Beamer slide example.

\documentclass[9pt]{beamer}
\usepackage[utf8]{inputenc}
\usetheme{inria}
\usepackage{helvet}
\author{John Doe}

\title[SHORTITLE]{the title \\[2mm] and its continuation}
\subtitle{subtitle}


% Automatically insert a "new section" page at each section.
\AtBeginSection[]{
  \begin{frame}[plain]
    \partpage
  \end{frame} 
}

% \inriaswitchcolors COLOR
%
% Where COLOR is one of red, blue, orange, darkblue, violet,
% pastelgreen, grey, or green.
\newcommand{\inriaswitchcolors}[1]{%
\pgfaliasimage{figfootline}{figfootline-#1}% !!!
\pgfaliasimage{figbackground}{figbackground-#1}% !!!
\pgfaliasimage{figbackground}{figbackground-#1}% !!!
}

\begin{document}

% starting the document
% *********************

% titlepage
% ---------
\begin{frame}[plain]
\titlepage
\end{frame}

% table of contents
% -----------------
\begin{frame}{\textcolor{inriaGrey}{Contents}}
  \tableofcontents
\end{frame}

% original template slides
% ************************
 
\section{Imported slides from the original template}

% Introduction slide
% ------------------

\begin{frame}{INTRODUCTION}
	Your text with scientific results or what ever... Your text with
scientific results or what ever... Your text with scientific results or
what ever... Your text with scientific results or what ever... Your
text with scientific results or what ever... Your text with scientific
results or what ever... Your text with scientific results or what
ever... Your text with scientific results or what ever... Your text
with scientific results or what ever...

Your text with scientific results or what ever... Your text with
scientific results or what ever... Your text with scientific results or
what ever... Your text with scientific results or what ever... Your
text with scientific results or what ever... Your text with scientific
results or what ever... Your text with scientific results or what
ever...
\end{frame}

% Default page in the template
% ----------------------------

\begin{frame}{Titre de la page}
Texte courant dunt am iriure commolut eumsandio odolor sectem
nonsed tinim nis nisl utem dion hent amcorer in vendio.

\begin{itemize}
\item Ugait delis aliquip sustie delit, secte velent aliquam, quisl ulla
  doloreet dolesed euguercip esent illa feugiam vent non henim.
\item To consed te dolesed uationsequat nonsed ex ea core dolore
velismo lendrer aesenim ea adiat.
\item Riurero conse modolorper se doloreetuero esequip.
  \begin{itemize}
  \item El iniamcom ea faccum nulputpat,
  \item Sequis adipism odolore dolent prat volore faccum venit,
  \item Consecte volorem inciliq uipsum et am accummo,
  \item Nummy nostin hent utpate ex.
  \end{itemize}
\end{itemize}
\end{frame}

% outer theme
% ***********

\inriaswitchcolors{blue}
\section{Testing the outertheme}

%
\subsection{Footline}
\begin{frame}
Goal: testing the footline.
\end{frame}
%
\subsection{Headline}
\begin{frame}{Frametitle}{Framesubtitle}
Goal: testing the frametitle.\\
Added Feature: Framesubtitle 
\end{frame}

% inner theme
% ***********

\inriaswitchcolors{green}
\section{Testing the innertheme}

%
\subsection{Table of contents}
\begin{frame}{\textcolor{inriaGrey}{Contents}}
  \tableofcontents[current,currentsubsection]
\end{frame}
%
\subsection{Itemize}
\begin{frame}{Itemize}
  \begin{itemize}
  \item item1
  \item item2
  \end{itemize}
\end{frame}
%
\subsection{Enumerations}
\begin{frame}{Enumerations}
  \begin{enumerate}
  \item <1->{item1}
  \item <2>{item2}
  \item <2->{item3}
  \end{enumerate}
\end{frame}
%
\subsection{Blocks}
\begin{frame}{Blocks}{Basics}
  \begin{block}{}
    a block without a title.
  \end{block}
  \begin{block}{a block with a title}
    some text.
  \end{block}
\end{frame}

\begin{frame}{Blocks}{normal,example and alert blocks}
  \begin{block}{A normal block}
  Text
 \end{block}

 \begin{alertblock}{An alert block}
  Text
 \end{alertblock}

 \begin{exampleblock}{An example block}
   Text
 \end{exampleblock}
\end{frame}

% color change
% ************

%
% To change color, you may do :
% \pgfaliasimage{<figname>}{<figname>-<color>}
%
% where <figname> is "figbackground" or "figfootline"
% and   <color>   is 
% red, blue, orange, darkblue, 
% violet, pastelgreen, grey or green.

% helper commands for this section
% --------------------------------
\newcommand{\testcolorfootline}[1]{%
\pgfaliasimage{figfootline}{figfootline-#1}% !!!
\begin{frame}{Testing color change}
  footline should be #1 here.
\end{frame}
}
%
\newcommand{\testcolorpartpage}[1]{%
\pgfaliasimage{figbackground}{figbackground-#1}% !!!
\begin{frame}[plain]
  \partpage
\end{frame}
}
%
\newcommand{\testcolortitlepage}[1]{%
\pgfaliasimage{figbackground}{figbackground-#1}% !!!
\begin{frame}[plain]
  \titlepage
\end{frame}
}


% the new section and its partpage 
\section{Testing color change}

% testing footline
\subsection{testing footline colors}
\testcolorfootline{blue}
\testcolorfootline{orange}
\testcolorfootline{darkblue}
\testcolorfootline{violet}
\testcolorfootline{pastelgreen}
\testcolorfootline{grey}
\testcolorfootline{green}

% testing partpage
\subsection{testing partpage colors}

\testcolorpartpage{blue}
\testcolorpartpage{orange}
\testcolorpartpage{darkblue}
\testcolorpartpage{violet}
\testcolorpartpage{pastelgreen}
\testcolorpartpage{grey}
\testcolorpartpage{green}

% testing titlepage
\subsection{testing titlepage colors}

\testcolortitlepage{blue}
\testcolortitlepage{orange}
\testcolortitlepage{darkblue}
\testcolortitlepage{violet}
\testcolortitlepage{pastelgreen}
\testcolortitlepage{grey}
\testcolortitlepage{green}


\end{document}
