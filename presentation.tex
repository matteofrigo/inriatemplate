%  Beamer slide example 
%
%%%
% Changelog
% - 2011-09-26 : v0.1 : import from a presentation + suppress some stuff
%%
\documentclass[slideopt,A4]{beamer}

\usepackage[utf8]{inputenc}
\usepackage{pgf}
\usepackage{fancyhdr}
\usepackage{multirow}
\usepackage{listings}
\usepackage{inria}
\input{beamerinnerthemeinria.sty}

\newcommand{\EMETTEUR}{Yohan Lee-tin-yien\ }
\newcommand{\talktitle}{ADT ParScaLi : Final report}
\newcommand{\talkdate}{September 13, 2011\ }
\newcommand{\inriafoot}{\EMETTEUR - \talktitle}


\begin{document}

\input{inria-atbeginsection.tex}



%   \setbeamertemplate{background canvas}[vertical shading][top=rouge1,bottom=rouge1,middle=rouge1]
%   \setbeamercolor{toto}{fg=blanc,bg=rouge1}
%   \setbeamertemplate{footline}
% {
% \begin{beamercolorbox}[wd=1\paperwidth,ht=15.5pt]{toto}
% \hspace{3mm}	\includegraphics[width=18mm]{logobasrouge1}
%   \hspace{1.5cm}  
%   \raisebox{2.5ex}
%   {EMETTEUR - NOM DE LA PRESENTATION}\hfill 
%   \raisebox{2.5ex}
%   {\today - \insertframenumber \hspace{5mm} \null }
% \end{beamercolorbox}}

% \begin{frame}
%
% \begin{textblock*}{10cm}(20mm,10mm)
% \textcolor{white}{\large INTRODUCTION}
% \end{textblock*}
%
% \begin{textblock*}{10cm}(20mm,20mm)
%
% 	Your text with scientific results or what ever... Your text with
% scientific results or what ever... Your text with scientific results or
% what ever... Your text with scientific results or what ever... Your
% text with scientific results or what ever... Your text with scientific
% results or what ever... Your text with scientific results or what
% ever... Your text with scientific results or what ever... Your text
% with scientific results or what ever...
%
% Your text with scientific results or what ever... Your text with
% scientific results or what ever... Your text with scientific results or
% what ever... Your text with scientific results or what ever... Your
% text with scientific results or what ever... Your text with scientific
% results or what ever... Your text with scientific results or what
% ever...
% \end{textblock*}
% \end{frame}






%%%%%%%%%%% Title page 

\setbeamertemplate{background canvas}[vertical shading][top=rouge1,middle=rouge1,bottom=rouge1]
\setbeamertemplate{footline}{\hspace{5em} \textcolor{white} {\null \hfill\talkdate}\hspace{2em}\null \vspace*{3pt}}

\begin{frame}

\begin{textblock*}{40mm}[0,0](10mm,0mm)
 \includegraphics[width=4cm]{logos/inria/logorouge1}
  \end{textblock*}

\begin{textblock*}{7cm}(13mm,50mm)
{\textcolor{white} {
{\huge ADT ParScaLi: }\\[2mm]
   { \huge Final report}\\[3mm]
   	{Parallel Scalable Hybrid Library for Large Scale Simulations} [Yohan Lee-tin-yien] }
	}
	\end{textblock*}


   \begin{textblock*}{40mm}[0,0](10mm,76mm)
  \begin{picture}(5,80)
\put(0,23){\includegraphics[width=4cm,height=1.5cm]{logos/inria/logobasrougeV1}}
\put(20,50){\tiny \textcolor{rouge2}{EQUIPE-PROJET}}
\put(20,43){\tiny \textcolor{rouge2}{HiePACS}}
\put(20,34){\tiny \textcolor{rouge2}{Centre Emetteur}}
\put(20,27){\tiny \textcolor{rouge2}{Bordeaux Sud-Ouest}}
\end{picture}
\end{textblock*}

\vspace*{-4pt}
\end{frame}
%%%%%%%%%%%%%%%%%%%%%%%%%%%%%
\setbeamertemplate{background canvas}[vertical shading][top=blanc, middle=blanc,bottom=blanc]
\setbeamertemplate{footline}
{
\begin{beamercolorbox}[wd=1\paperwidth,ht=15.5pt]{toto}
\hspace{3mm}	
\includegraphics[width=17mm]{logos/inria/logobastrans}
  \hspace{1.5cm}  
  \raisebox{2.5ex}
  {\inriafoot}\hfill 
  \raisebox{2.5ex}
  {\talkdate - \insertframenumber \hspace{5mm} \null }
\end{beamercolorbox}}
%%%%%%%%%%%%%%%%%%%%%%%%%%
\begin{frame}
  \frametitle{\bfseries Outline}
  \tableofcontents[hideallsubsections]
\end{frame}
%%%%%%%%%%%%%%%%%%%%%%%%%%%

\setbeamercolor{toto}{fg=blanc,bg=rouge1}

%%
\section{Introduction}
%%
\subsection{Motivations}
\frame{\frametitle{Motivations} 
\centerline{Goal: solving ${\cal A}x =b$, where ${\cal A}$ is sparse}
}
\subsection{Hybrid method}
%%
%%
\begin{frame}{Hybrid method}{Step by step}
\begin{minipage}{0.5 \textwidth}
  \begin{enumerate}
  \item \only<1>{Partition the initial system}
  \item \only<2>{Factorize the local system And
    Get the Schur complement}
  \item \only<3>{Compute the preconditioner}
  \item \only<4>{Solve the system}
  \end{enumerate}
\end{minipage}
\hskip 0.63cm
\begin{minipage}{0.40\textwidth}
    \only<1>{
      \centerline{Use: a partitioner}
      \centerline{\alert{METIS} or \alert{SCOTCH}}
      {\small
	   $$ {\cal A}^{(i)} \equiv
	  \begin{pmatrix}
	      {\cal A}_{{\cal I}_i{\cal I}_i} 		& {\cal A}_{{\cal I}_i\Gamma_i}	 		 \cr
	      {\cal A}_{{\cal I}_i\Gamma_i} 		& {\cal A}_{\Gamma\Gamma}^{({i})}	 		 \cr
	  \end{pmatrix}
           $$
	   }

      }
    \only<2>{
      \centerline{Use: a sparse direct solver}
      \centerline{\alert{MUMPS} or \alert{PaSTiX}}
      $$
      \alert{{\cal S}^{({i})} = {\cal A}_{\Gamma\Gamma}^{({i})} - 
        {\cal A}_{\Gamma_{i} {\cal I}_{i}} {\cal A}_{{\cal I}_{i}{\cal I}_{i}}^{-1} {\cal A}_{{\cal I}_{i} \Gamma_{i}}}
      $$
      \small{Focus on the Schur system:}
      ${\cal S} x_{\Gamma} =f$
      }
    \only<3>{
      \begin{enumerate}
      \item Copy \& assemble 
        $$\alert{{\cal S}^{({i})}}$$
    \item Factorize it or its approx.
      (\alert{LAPACK}, \alert{MUMPS} or \alert{PaSTiX})
    \end{enumerate}
    }
    \only<4>{
      \begin{enumerate}
      \item Compute the second members
      \item Solve iteratively (\alert{GMRES/CG})
        ${\cal S} x_{\Gamma} = f$
      \item Solve lc. system (\alert{MUMPS} or \alert{PaSTiX})
        $${\cal A}_{{\cal I}_i{\cal I}_i}x_{{\cal I}_i} = b_{{\cal I}_i} - {\cal A}_{{\cal I}_i\Gamma_i}x_{\Gamma_{i}}$$
      \item Gather the solution
      \end{enumerate}
    }
\end{minipage}
\end{frame}

%
\subsection{ADT}
\begin{frame}{ADT}{Context}
\begin{itemize}
\item Main idea: mix the iterative and direct concepts 
$\Rightarrow$ Hybrid method

\item PHD + postdoctoral (4.5 years): Prototype Solver called MaPHyS
\item MaPHyS: Massively Parallel Hybrid Solver
\end{itemize}
\end{frame}

\begin{frame}{ADT}{Objectives}

MaPHyS - prototype $\rightarrow$ stable library 

\begin{itemize}
\item<1-> Improve code (conform to standards, add structures !)
\item<2-> Add genericity ($3^{rd}$ party softwares)
\item<3-> Validate + improvement performances
\item<4-> Write user and dev guide
\end{itemize}

\end{frame}

%
\subsection{Team}
\begin{frame}{Team}

HiePACS:
  \begin{itemize}
  \item Jean Roman
  \item Luc Giraud
  \item Emmanuel Agullo
  \item Abdou Guermouche
  \item Yohan Lee-tin-yien
  \end{itemize}
\pause
People:
  \begin{itemize}
  \item Hervé Mathieu (SED)
  \item Azzam Haidar (UTK)
  \end{itemize}
\end{frame}

%
\subsection{Organization}
\begin{frame}{Organization}
\begin{itemize}
\item Weekly "what was done" [wiki]
\item Weekly "what is expected" [wiki]
\item Weekly phone meetings
\pause
\item Monthly IJD meetings w/ Hervé Mathieu (SED)
\pause
\item Punctuals meetings 
  \begin{itemize}
  \item Luc Giraud's passage
  \item Missions to Toulouse CERFACS (technical talks) with Luc Giraud or Azzam Haidar
  \item Quarterly meetings
  \end{itemize}
\pause
\item Links
  \begin{itemize}
  \item Wiki [internal]: https://wiki.bordeaux.inria.fr/maphys/doku.php
  \item Gforge [private]: https://gforge.inria.fr/projects/maphys/
  \item Doxygen docs [internal]: http://maphys.bordeaux.inria.fr/
  \end{itemize}
\end{itemize}


\end{frame}

%
\section{MaPHyS' source code}
\subsection{Prototype's source code}
\begin{frame}{Prototype's source code}
\begin{minipage}{0.42 \textwidth}
\begin{itemize}
  \item No structures
  \item Hence, lack of genericity
  \item Old coding style (FORTRAN 77)
  \item No user guide, few comments
  \item Initial API, use of shared variable 
\end{itemize}
\only<2->{\alert{$\Rightarrow$ Rewrite the code}}
\end{minipage}
\end{frame}
%
\subsection{Rewritten source code}
\begin{frame}{Rewritten source code}
\begin{minipage}{0.60 \textwidth}
\begin{itemize}
  \item Use structures
  \item Genericity added 
  \item Follow F90 coding style recommendations
  \item Add user guide and a lot of comments
  \item Better multi-arithmetic handling
  \item New API, one structure based. Similar to well-known F90 solvers (Harwell Boeing, MUMPS).
\end{itemize}
\end{minipage}
\end{frame}
%

%
\subsection{Compilation}
\begin{frame}{Compilation}
  Requested Features \& issues:
  \begin{itemize}
  \item Portable, standard
  \item Support of multi-arithmetic
  \item Future support of mixed arithmetic % $\Rightarrow$ CPP not enough
    (single/double precisions)
  \end{itemize}
  
  Solution:
  \begin{itemize}
  \item GNU/Make 
  \item Use a custom script
  \end{itemize}

  % Example: 
  % \only<2->{
  % \only<2>{{\bf x}mph\_maphys\_mod.}\only<3->{{\bf c}mph\_maphys\_mod.}\only<2-3>{\bf F90}\only<4>{\bf f90}\only<5>{\bf o}  \only<1>{(generic)}
  % \only<3>{\bf [script]}
  % \only<4>{\bf [cpp]}
  % \only<5>{\bf [Fortran 90 compiler]}
  % }
\end{frame}
%
\subsection{New features}
\begin{frame}{New features}
   \begin{itemize}
     \item Enabling "MPI + Threads" with multithreaded {\bf PaSTiX}
          \begin{itemize}
           \item Use MPI between domains
           \item Use Threads inside domain 
          \end{itemize}
     \item Support of Symmetric/SPD systems.
       \begin{itemize}
       \item Use dedicated sparse direct solver options
       \item Use dedicated LAPACK routines
       \item Use {\bf PackCG} (Conjugate Gradient)
       \end{itemize}
     \item Genericity in partitioning
       \begin{itemize}
       \item Interface {\bf SCOTCH} software 
       \item Plug a distributed generator (distributed inputs)
       \end{itemize}
     \item Statistics (performances, memory usages) 
   \end{itemize}
\end{frame}
%%
%%
%
\subsection{New API}
%%
\section{MaPHyS' validations}
\subsection{Overview of validations}
\begin{frame}{Overview of validations}
   \begin{itemize}
     \item Abstraction or components \only<2->{{: \bf unit tests}} \only<5->{{, \bf every commit }}
     \item Functionality or strategy \only<3->{{: \bf elemental cases}} \only<6->{{, \bf every feature }}
     \item Performances \only<4->{: \bf average cases} \only<7->{{, \bf every benchmark }}
   \end{itemize}
\end{frame}

\subsection{Example: description}
\frame{\frametitle{Example: description}
\only<1>{
  \begin{block}{{\sc Audi - PARASOL}}
    \begin{itemize}
    \item {3D structural mechanics problem}
    \item {943,695 dof}
    \item {39,297,771 nnz}  
    \end{itemize}
  \end{block}
}
\only<2>{
    \begin{textblock*}{\textwidth}(22mm,15mm)
      { \tiny
      \begin{block}{{\sc Loulou}}
        \begin{itemize}
        \item 1 node \phantom{()}
        \item Dual-socket hexa-core CPU
        \item Intel Nehalem \\ X5670 @ 2.93 GHz
        \item RAM: 96 GB 
        \item Ethernet \\@ 1 Gb/s
        \end{itemize}
      \end{block}
      
      \begin{block}{{\sc Fourmi}}
        \begin{itemize}
        \item 68 nodes (up to 16)
        \item Dual-socket quad-core CPUs
        \item Intel Nehalem \\ X5550 @ 2.66 GHz
        \item RAM: 24 GB/node
        \item Infiniband QDR \\ @ 40 Gb/s
     \end{itemize}
  \end{block}
}
\end{textblock*}
}
}

\subsection{Example: performances}
\begin{frame}{Example: performances}
\centerline{\bf Issue: Schur assembly is slow}
\centerline{\bf Origin: MPI\_Bsend performances are variables.}
\centerline{\bf Solution: use MPI\_Isend instead}
\end{frame}
%%
%%
\begin{frame}{Workshops \& conferences}
\subsection{Workshops \& conferences}
 Parallelism specific:
 \begin{itemize}
   \item NCSA-INRIA 2010 [Bordeaux, 2010]
   \item EUROPAR 2011 [Bordeaux, 2011]
 \end{itemize}

 Math specific:
 \begin{itemize}
   \item SPARSE DAYS 2010 [Toulouse, 2010]
   \item PRECOND 2011 [Bordeaux, 2011]
 \end{itemize}
\end{frame}

\begin{frame}{Trainings \& acquired skills }
\subsection{Trainings \& acquired skills }
 Trainings:
 \begin{itemize}
   \item Software Project Management training
   \item GPGPU training
   \item Technical trainings by Bordeaux' SED
 \end{itemize}

 Skills:
 \begin{itemize}
   \item Tools: CMake, Autotools, Buildbot
   \item Languages: Python, Perl, Fortran 90
   \item Standards: Fortran 90 \& Compilation standards
 \end{itemize}
\end{frame}


%%
\section{Conclusion}
\begin{frame}{Conclusion}
For Maphys:
\begin{itemize}
  \item Improved code quality
  \item Currently under performances validations
  % \item 3 months of a future PhD student
  % \item Future PhD on MaPHyS 2.0 planned...
  \item Solid platform for future research (GPUs, ...)  
\end{itemize}

\only<2->{
For me: 
\begin{itemize}
  \item Great experience (Workshops,Confs,Skills)
  % \item Future ? \only<3->{ Find a new job}
\end{itemize}
}
\end{frame}
%
%
\setbeamertemplate{background canvas}[vertical shading][top=rouge1,middle=rouge1,bottom=rouge1]
\setbeamertemplate{footline}{}
\begin{frame}
\begin{center} 
\textcolor{white} {\huge Questions ?}
\end{center}

\begin{textblock*}{7cm}(60mm,76mm)
{\textcolor{white} {
{Bordeaux}\\[1mm]
   {INRIA Bordeaux Sud-Ouest}\\[2mm]
   	{www.inria.fr}}
	}
	\end{textblock*}

   \begin{textblock*}{40mm}[0,0](10mm,76mm)
  \begin{picture}(5,80)
\put(0,23){\includegraphics[width=4cm]{logos/inria/logobasrouge1}}

\end{picture}
\end{textblock*}

\vspace*{-4pt}
\end{frame}


\end{document}


